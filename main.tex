\documentclass[sigconf,nonacm]{acmart}

%% Enable subfigures
\usepackage{subfigure}
%% Enable numbers in scientific format.
\usepackage{siunitx}
%% Enable enumerate start from.
\usepackage{enumitem}

%% Enable theorems
\newtheorem{theorem}{Theorem}[section]
\newtheorem{lemma}[theorem]{Lemma}

%% Enable algorithms
\usepackage{algorithm}
\usepackage[noend]{algpseudocode}
\let\ReturnInline\Return
\renewcommand{\Return}{\State\ReturnInline}
\algrenewcommand\algorithmicrequire{$\rhd$}
\algrenewcommand\algorithmicensure{$\square$}

%% Fonts used in the template cannot be substituted; margin 
%% adjustments are not allowed.
\AtBeginDocument{%
  \providecommand\BibTeX{{%
    \normalfont B\kern-0.5em{\scshape i\kern-0.25em b}\kern-0.8em\TeX}}}

%% Rights management information.
\setcopyright{acmcopyright}
\copyrightyear{2018}
\acmYear{2018}
\acmDOI{XXXXXXX.XXXXXXX}

%% These commands are for a PROCEEDINGS abstract or paper.
\acmConference[Conference acronym 'XX]{Make sure to enter the correct
  conference title from your rights confirmation emai}{June 03--05,
  2018}{Woodstock, NY}
%% Title of the proceedings is different from ``Proceedings of ...''?
% \acmBooktitle{Woodstock '18: ACM Symposium on Neural Gaze Detection,
%  June 03--05, 2018, Woodstock, NY} 
% \acmPrice{15.00}
% \acmISBN{978-1-4503-XXXX-X/18/06}

%% Submission ID.
% \acmSubmissionID{123-A56-BU3}

%% Use the "author year" style of citations and references?
% \citestyle{acmauthoryear}

%% Message
\newcommand{\kk}[1]{{{\color{red} #1}}}
\newcommand{\ds}[1]{{{\color{blue} #1}}}
\newcommand{\su}[1]{{{\color{green} #1}}}

%% Ignore block
\newcommand{\ignore}[1]{}

%% Macros
\newcommand{\Lou}{\textit{Louvain}}
\newcommand{\LPA}{\textit{LPA}}
\newcommand{\Hyb}{\textit{Hybrid Louvain-LPA}}
\newcommand{\Sta}{\textit{Static}}
\newcommand{\Nai}{P-ND}
\newcommand{\DelOrg}{\textit{$\Delta$-screening}}
\newcommand{\Del}{P-DDS}
\newcommand{\Fro}{P-DF}
\newcommand{\StaLou}{\textit{Static Louvain}}
\newcommand{\NaiLou}{$\text{P-ND}_\text{L}$}
\newcommand{\DelLou}{$\text{P-DDS}_\text{L}$}
\newcommand{\FroLou}{$\text{P-DF}_\text{L}$}
\newcommand{\StaLPA}{\textit{Static LPA}}
\newcommand{\NaiLPA}{$\text{P-ND}_\text{LPA}$}
\newcommand{\DelLPA}{$\text{P-DDS}_\text{LPA}$}
\newcommand{\FroLPA}{$\text{P-DF}_\text{LPA}$}
\newcommand{\FroHyb}{$\text{P-DF}_\text{H}$}




\begin{document}

%% Full title of the paper.
\title[GVE-LPA: Fast Label Propagation Algorithm (LPA) for Community Detection in Shared Memory Setting]{GVE-LPA: Fast Label Propagation Algorithm (LPA) for Community Detection in Shared Memory Setting}

%% Short title to be used in page headers (optional).
% \title[short title]{full title}
% \subtitle{Something other than the title}

%% Authors and their affiliations.
\author{Subhajit Sahu}
\email{subhajit.sahu@research.iiit.ac.in}
\affiliation{%
  \institution{IIIT Hyderabad}
  \streetaddress{Professor CR Rao Rd, Gachibowli}
  \city{Hyderabad}
  \state{Telangana}
  \country{India}
  \postcode{500032}
}

%% Concise author list in page headers.
%\renewcommand{\shortauthors}{Sahu, Kothapalli, and Banerjee, et al.}

%% Show page numbers.
\settopmatter{printfolios=true}

%% Short summary of the work to be presented in the article.
\begin{abstract}
Community detection is the problem of identifying natural divisions in networks. Efficient parallel algorithms for this purpose are crucial in various applications, particularly as datasets grow to substantial scales. This technical report presents an optimized parallel implementation of the Label Propagation Algorithm (LPA), a high speed community detection method, for shared memory multicore systems. On a server equipped with dual 16-core Intel Xeon Gold 6226R processors, our LPA, which we term as GVE-LPA, outperforms FLPA, igraph LPA, and NetworKit LPA by $139\times$, $97000\times$, and $40\times$ respectively - achieving a processing rate of $1.4 B$ edges/s on a $3.8 B$ edge graph. In addition, GVE-LPA scales at a rate of $1.7\times$ every doubling of threads.
\end{abstract}

%% The code below is generated by the tool at http://dl.acm.org/ccs.cfm.
\begin{CCSXML}
<ccs2012>
<concept>
<concept_id>10003752.10003809.10010170</concept_id>
<concept_desc>Theory of computation~Parallel algorithms</concept_desc>
<concept_significance>500</concept_significance>
</concept>
<concept>
<concept_id>10003752.10003809.10003635</concept_id>
<concept_desc>Theory of computation~Graph algorithms analysis</concept_desc>
<concept_significance>500</concept_significance>
</concept>
</ccs2012>
\end{CCSXML}

% \ccsdesc[500]{Theory of computation~Parallel algorithms}
% \ccsdesc[500]{Theory of computation~Graph algorithms analysis}

%% Pick words that accurately describe the work being presented.
\keywords{Community detection, Parallel Label Propagation Algorithm (LPA)}

% \received{20 February 2007}
% \received[revised]{12 March 2009}
% \received[accepted]{5 June 2009}



%% Process the author and title information.
\maketitle

\section{Introduction}
\label{sec:introduction}
Community detection\ignore{, also know as clustering,} is the problem of uncovering the underlying structure of complex networks, i.e., identifying groups of vertices that exhibit dense internal connections but sparse connections with the rest of the network, in an unsupervised manner. It is an NP-hard problem with numerous applications in domains such as drug discovery, protein annotation, topic discovery, anomaly detection, and criminal identification. Communities identified are intrinsic when based on network topology alone, and are disjoint when each vertex belongs to only one community \cite{com-gregory10}.\ignore{One of the difficulties in the community detection problem is the lack of apriori knowledge on the number and size distribution of communities \cite{com-blondel08}.} The \textit{Label Propagation Algorithm (LPA)}, also known as RAK, \cite{com-blondel08} is a popular heuristic-based approach for community detection, with the modularity metric \cite{com-newman06} being used to measure the quality of communities identified.

In recent years, the collection of data and the relationships among them, represented as graphs, have reached unmatched levels. This has necessitated the design of efficient parallel algorithms for community detection on large networks. The multicore/shared memory setting is crucial for community detection due to its energy efficiency and the prevalence of hardware with extensive DRAM sizes.\ignore{Optimizing parallel community detection algorithms for modern hardware architectures can yield notable performance benefits and competitive advantages across applications.} However, many of the current algorithms for community detection are challenging to parallelize due to their irregular and inherently sequential nature \cite{com-halappanavar17}, in addition to the complexities of handling concurrency, optimizing data access, reducing contention, minimizing load imbalance. Existing studies on LPA propose algorithmic optimizations and parallelization techniques, but do not address optimization for the aggregation phase of the LPA algorithm, which emerges as a bottleneck after the local-moving phase of the algorithm has been optimized. Moreover, these optimization techniques are scattered over a number of papers, making it difficult for a reader to get a grip over them.




\subsection{Our Contributions}

This report introduces GVE-LPA\footnote{https://github.com/puzzlef/rak-communities-openmp}, an optimized parallel implementation of LPA for shared memory multicores. On a machine with two 16-core Intel Xeon Gold 6226R processors, GVE-LPA outperforms FLPA, igraph LPA, and NetworKit LPA by $97,000\times$, $118,000\times$, and $40\times$ respectively, achieving a processing rate of $1.4 B$ edges/s on a $3.8 B$ edge graph. With doubling of threads, GVE-LPA exhibits an average performance scaling of $1.7\times$.




%% - Use --- for a dash.
%% - Use ``camera-ready'' for quotes.
%% - Use {\itshape very} or \textit{very} for italicized text.
%% - Use \verb|acmart| or {\verb|acmart|} for mono-spaced text.
%% - Use \url{https://capitalizemytitle.com/} for URLs.
%% - Use {\bfseries Do not modify this document.} for important boldface details.
%% - Use \ref{fig:name} for referencing.

%% For a block of pre-formatted text: 
% \begin{verbatim}
%   \renewcommand{\shortauthors}{McCartney, et al.}
% \end{verbatim}

%% For a list of items:
% \begin{itemize}
% \item the ``ACM Reference Format'' text on the first page.
% \item the ``rights management'' text on the first page.
% \item the conference information in the page header(s).
% \end{itemize}

%% For a table:
% \begin{table}
%   \caption{Frequency of Special Characters}
%   \label{tab:freq}
%   \begin{tabular}{ccl}
%     \toprule
%     Non-English or Math&Frequency&Comments\\
%     \midrule
%     \O & 1 in 1,000& For Swedish names\\
%     $\pi$ & 1 in 5& Common in math\\
%     \$ & 4 in 5 & Used in business\\
%     $\Psi^2_1$ & 1 in 40,000& Unexplained usage\\
%   \bottomrule
% \end{tabular}
% \end{table}

%% For a full-width table:
% \begin{table*}
%   \caption{Some Typical Commands}
%   \label{tab:commands}
%   \begin{tabular}{ccl}
%     \toprule
%     Command &A Number & Comments\\
%     \midrule
%     \texttt{{\char'134}author} & 100& Author \\
%     \texttt{{\char'134}table}& 300 & For tables\\
%     \texttt{{\char'134}table*}& 400& For wider tables\\
%     \bottomrule
%   \end{tabular}
% \end{table*}


%% For inline math:
% \begin{math}
%   \lim_{n\rightarrow \infty}x=0
% \end{math},

%% For a numbered equation:
% \begin{equation}
%   \lim_{n\rightarrow \infty}x=0
% \end{equation}

%% For an unnumbered equation:
% \begin{displaymath}
%   \sum_{i=0}^{\infty} x + 1
% \end{displaymath}

%% For a figure:
% \begin{figure}[h]
%   \centering
%   \includegraphics[width=\linewidth]{inc/sample-franklin}
%   \caption{1907 Franklin Model D roadster. Photograph by Harris \&
%     Ewing, Inc. [Public domain], via Wikimedia
%     Commons. (\url{https://goo.gl/VLCRBB}).}
%   \Description{A woman and a girl in white dresses sit in an open car.}
% \end{figure}

%% For a teaser figure.
% \begin{teaserfigure}
%   \includegraphics[width=\textwidth]{sampleteaser}
%   \caption{figure caption}
%   \Description{figure description}
% \end{teaserfigure}


\section{Related work}
\label{sec:related}
The \textit{Label Propagation Algorithm (LPA)} is a diffusion-based technique for identifying communities. It is faster and more scalable than Louvain, as it does not require repeated optimization steps and is easier to parallelize \cite{com-newman04, com-raghavan07}. For parallelization of LPA, vertex assignment has been achieved with guided scheduling \cite{staudt2015engineering}, parallel bitonic sort \cite{soman2011fast}, and pre-partitioning of the graph \cite{kuzmin2015parallelizing}.

Additional improvements upon the LPA include using a stable (non-random) mechanism of label choosing in the case of multiple best labels \cite{com-xing14}, addressing the issue of monster communities \cite{com-berahmand18, com-sattari18}, identifying central nodes and combining communities for improved modularity \cite{com-you20}, and using frontiers with alternating push-pull to reduce the number of edges visited and improve solution quality \cite{com-liu20}. A number of variants of LPA have been proposed, but the original formulation is still the simplest and most efficient \cite{garza2019community}.

A few open source implementations and software packages have been developed for community detection using LPA. Fast Label Propagation Algorithm (FLPA) \cite{traag2023large} is a fast variant of the LPA, which utilizes a queue-based approach to process only vertices with recently updated neighborhoods. NetworKit \cite{staudt2016networkit} is a software package designed for analyzing the structural aspects of graph data sets with billions of connections. It is implemented as a hybrid with C++ kernels and a Python frontend, and includes parallel implementation of LPA. igraph \cite{csardi2006igraph} is a similar package, written in C, with Python, R, and Mathematica frontends. It is widely used in academic research, and includes implementation of LPA.

We now discuss the implementation of LPA in NetworKit, igraph, and FLPA. The parallel implementation of LPA of NetworKit is given in \texttt{NetworKit::PLP::run()} ...


%% How NetworKit PLP works
- In the function NetworKit::PLP::run()
- Starts with a unique label for each node
- Done with a plain OpenMP for (1 chunk size, false sharing)
- They use threshold heuristic, with threshold = $N/10^5$
- They use a vector-bool for tracking active nodes (vertex pruning opt.)
- They use the guided-schedule parallel for to loop through active nodes
- They use a std-map for storing the label weights, created for each vertex (SLOW!!)
- They use atomic ops. to count the number of updated vertices


%% How igraph LPA works
- In the function igraph\_community\_label\_propagation()
- Shuffles node order in each iteration
- Keeps track of dominant labels for each new neighbor of a vertex
- Selects one of the dominant labels using a random number generator
- Clears hashtable by going through neigh neighbor
- Uses same hashtable as ours
- Does not use pruning optimization
- Switches between label updating and control iterations
- Checks if current label of node is not dominant in control iteration
- No tolerance


%% How igraph FLPA works
- In the function igraph\_i\_community\_label\_propagation()
- It uses a dequeue for a set of unprocessed vertices, if empty then converged
- It does not shuffle for random node order
- If label changed, FLPA considers neighbors not in the same community








% https://www.sciencedirect.com/science/article/abs/pii/S0378437119312026
% http://proceedings.mlr.press/v32/fujiwara14.pdf
\ignore{
A number of variants of LPA have been proposed \cite{wang2015hlpa, zhang2016social, ma2018psplpa, joghan2019weighted}.




Label propagation is an effective and efficient technique to utilize local and global features in a network for semi-supervised learning \cite{hwang2010heterogeneous}.

Simply propagating the label on the combined network is not a principled method to explore the cluster structures in the network since each homo-subnetwork may have its own cluster structure and each hetero-subnetwork may also have its own bicluster structures \cite{hwang2010heterogeneous}. Hwang and Kuang \cite{hwang2010heterogeneous} introduce MINProp (Mutual Interaction-based Network Propagation, a regularization framework for propagating information between subnetworks in a heterogeneous network. MINProp sequentially performs label propagation on each individual subnetwork with the current label information derived from the other subnetworks and repeats this step until convergence to the global optimal solution to the convex objective function of the regularization framework. The independent label propagation on each subnetwork explores the cluster structure in the subnetwork. Hwang and Kuang \cite{hwang2010heterogeneous} apply the MINProp algorithm to disease gene discovery from a heterogeneus network of disease phenotypes and genes. MINProp discovered new disease-gene associations that are only reported recently.

Wang et al. \cite{wang2013label} use label propagation to propagate labels from massive 2D semantic labeled datasets, such as ImageNet, to 3D point clouds, due to the difficulty in acquiring sufficient 3D point labels towards training effective classifiers.

Label propagation has been shown to be effective in many automatic segmentation applications \cite{wang2014geodesic}.

Zhang et al. \cite{zhang2017label} propose LPA\_NI, a label propagation algorithm for community detection based on node importance and label influence. Their algorithm first evaluates the importance of each node, and processes the nodes based on descending of their importance. Further, when updating the label of a node, if multiple labels are maximally popular, Zhang et al. calculate the influence of each label, and select the most influential label. Both of these techniques help improve the stability of LPA. However, they rely on priori knowledge to calculate the importance of each node, and the influence of each label.

Zheng et al. \cite{zheng2018improved} introduce a metric, called label purity, which represents the edge-weighted percentage of neighbors which have the same label as the given node. The purity of a node is initialized as being inversely proportional its degree, based on the idea that a vertex with more neighbors is less likely to preserve its own label. To optimize for execution time, Zheng et al. change the processing order of nodes, by processing nodes with higher weighted-degree first. Further, they apply an attenuation factor to attenuate later iterations of label propagation, based on the idea that a label's update time can approximately describe the label's source vertex and the updated vertex, and that signals attenuate with distance. Since, this reduce the probability of label update in later iterations, it allows their algorithm to converge faster.

A similar approach has been propsed by Sattari and Zamanifar \cite{sattari2018spreading}, which attempts to address the issue of monster communities identified LPA, which decrease the quality of communities identified by it. Their method assigns an activation value to each label, and label-activation pairs are propagated by the algorithm.

The most commonly used method to tackle the graph partitioning problem in practice is the multilevel approach. During a coarsening phase, a multilevel graph partitioning algorithm reduces the graph size by iteratively contracting nodes and edges until the graph is small enough to be partitioned by some other algorithm. A partition of the input graph is then constructed by successively transferring the solution to the next finer graph and applying a local search algorithm to improve the current solution \cite{meyerhenke2014partitioning}.

Meyerhenke et al. \cite{meyerhenke2014partitioning} present an approach to partition graphs by by iteratively contracting size-constrained clusterings that are computed using a label propagation algorithm. They also use the same algorithm for uncoarsening as a fast and simple local search algorithm.

Gottesburen et al. \cite{gottesburen2021scalable} present Mt-KaHyPar, the first shared-memory multilevel hypergraph partitioner with a parallel coarsening algorithm that uses parallel community detection as guidance, initial partitioning via parallel recursive bipartitioning with work-stealing, a scalable label propagation refinement algorithm, and the first fully-parallel direct k-way formulation of the classical FM algorithm \cite{fiduccia1988linear}.  If a partition is still imbalanced, on the finest level, Gottesburen et al. rebalance it using an approach that is similar to label propagation.

Parkway \cite{trifunovic2004k}, Mt-KaHIP \cite{akhremtsev2020high}, and ParHiP \cite{meyerhenke2017parallel} use size-constrained label propagation.

Bond percolation is quite similar to the label propagation in community detection \cite{peng2014accelerating}.

Label propagation is very fast \cite{peng2014accelerating}. This is a fast method that simply updates the label of a node according to the plurality vote of its neighbors. It is appropriate for large networks, though the quality is usually compromised. The label propagation is extremely efficient, so it is hard to improve its run time for small graphs\cite{peng2014accelerating}. We also note that label propagation has the lowest score with respect to modularity but the highest score w.r.t. NMI compared to ground truth \cite{peng2014accelerating}.

Since computing the pairwise similarity between the training data is prohibitively expensive in most kinds of input data, currently, there is no general ready-to-use semi-supervised learning method/tool available for learning with tens of millions or more data points \cite{petegrosso2017low}.

Xu et al. \cite{xu2019distributed} propose a distributed temporal link prediction algorithm based on label propagation (DTLPLP). Here, nodes are associated with labels, which include details of their sources, and the corresponding similarity value. When such labels are propagated across neighbouring nodes, they are updated based on the weights of the incident links, and the values from same source nodes are aggregated to evaluate the scores of links in the predicted network.

Soman and Narang \cite{soman2011fast} present the design of a parallel GPU-based algorithm for community detection using weigthed label propagation algorithm.

Zhang et al. \cite{zhang2023large} propose a community detection method based on core node and layer-by-layer label propagation, and extend it to identifying overlapping communities. First, they identify core nodes whose node degree is greater than the average degree in the graph. Label propagation is then carried out layer-by-layer, starting from the core nodes.

Ma et al. \cite{ma2018psplpa} propose a label propagation algorithm on Spark, called PSPLPA (Probability and similarity based Parallel label propagation algorithm).

Precision of link prediction can be improved to a great extent by including community information in the prediction methods \cite{mohan2017scalable}.

Mohan et al. \cite{mohan2017scalable} propose a scalable method for community structure-based link prediction on large networks. This method uses a parallel label propagation algorithm for community detection and a parallel community information-based Adamic–Adar measure for link prediction.

Li et al. \cite{li2015detecting} use a synchronous implementation of LPA.

Zhang et al. \cite{zhang2020lilpa} propose Label Importance based Label Propagation Algorithm (LILPA), is proposed to discover communities by adopting fixed label update order based on the ascending order of node importance. They also apply LILPA in a drug network to discover drug communities and core drugs for treating different indications in Traditional Chinese Medicine (TCM).

A multilevel method is a scalable strategy to solve optimization problems in large bipartite networks, which operates in three stages. Initially the input network is iteratively coarsened into a hierarchy of gradually smaller networks. Coarsening implies in collapsing vertices into so-called super-vertices which inherit properties of their originating vertices. An initial solution is obtained executing the target algorithm in the coarsest network. Finally, this solution is successively projected back over the inverse sequence of coarsened networks, up to the initial one, yielding an approximate final solution \cite{valejo2020coarsening}.

Valejo et al. \cite{valejo2020coarsening} introduce a weight-constrained variation of label propagation for fast coarsening of graphs --- that allows users to specify the desired size of the coarsest network and control super-vertex weights.

Maleki et al. \cite{maleki2020dhlp} present two distributed label propagation algorithms for heterogeneous networks.

Roghani et al. \cite{roghani2021pldls} propose a Spark-based Parallel Label Diffusion and Label Selection (PLDLS) based community detection algorithm. In the first phase of their algorithm, they identify nodes forming triangles as core nodes, assign labels based on the idea that nodes forming triangles tend to be in the same community, and diffuse labels up to two levels. In the second phase, they perform the iterative process of LPA.

Zarei et al. \cite{zarei2020detecting} propose an algorithm called Weighted Label Propagation Algorithm (WLPA) to detect community structure in signed and unsigned social networks. In WLPA, first, the similarity of all adjacent nodes is estimated by using MinHash. Then, each edge is assigned a
weight equal to the estimated similarity of its end nodes. The weights assigned to the edges somehow indicate the intensity of communication
between users. Finally, the community structure of the network is determined through the weighted label propagation.

Shen and Yang \cite{shen2016topic} present simLPA, which fuses the content-based and link-structure based approaches for community detection.

El Kouni et al. \cite{el2021wlni} propose an algorithm, called WLNI-LPA, based on label propagation for detecting efficient community structure in the attributed network. WLNI-LPA is an extension of LPA that combines node importance, attributes information, and topology structure to improve the quality of graph partition.

Farnadi et al. \cite{farnadi2015scalable} propose Adaptive Label Propagation, which dynamically adapts to the underlying characteristics of homophily, heterophily, or otherwise, of the connections of the network, and applies suitable label propagation strategies accordingly.

Berahmand and Bouyer \cite{berahmand2018lp} propose Label influence Policy for Label Propagation Algorithm (LP-LPA), which measures link strength value for edges and nodes’ label influence value for nodes in a new label propagation strategy with preference on link strength and for initial nodes selection, avoid of random behavior in tiebreak states, and efficient updating order and rule update. These procedures can sort out the randomness issue in an original LPA and stabilize the discovered communities in all runs of the same network.

Li et al. \cite{li2015parallel} propose Parallel Multi-Label Propagation Algorithm (PMLPA), which employs a new label updating strategy using ankle-value in the label propagation procedure during each iteration, to detect the overlapping communities in networks.

Vitali \cite{henne2015label} apply label propagation to hypergraph clustering. He evaluates three adaptations of label propagation as coarsening strategies in a direct k-way multilevel hypergraph partitioning framework. Vitali also propose a greedy local search algorithm inspired by label propagation for the uncoarsening and refinement phase of the multilevel partitioning heuristic.

Ye et al. \cite{ye2023large} propose GLP, a GPU-based framework to enable efficient LP processing on large-scale graphs.

Boldi et al. \cite{boldi2011layered} propose Layered Label Propagation, which uses clusterings of nodes in various layers to reorder nodes in the graph. These can then be compressed with the WebGraph compression framework \cite{boldi2004webgraph}.

Meyerhenke et al. \cite{meyerhenke2016partitioning} introduce Size-Constrained Label Propagation (SCLaP) and show how it can be used to instantiate both the coarsening phase and the refinement phase of multilevel graph partitioning.

Zhang et al. \cite{zhang2020multilevel} propose a multilevel partition algorithm based on weighted label propagation.

Slota et al. \cite{slota2020scalable} introduce XtraPuLP, a distributed-memory graph partitioner based on LPA. Their implementation can be generalized to compute partitions with an arbitrary number of constraints, and it can compute partitions with balanced communication load across all parts.

Wang et al. \cite{wang2014partition} propose a Multilevel Label Propagation (MLP) method for graph partitioning.

Akhremtsev et al. \cite{akhremtsev2020high} present an approach to multi-level shared-memory parallel graph partitioning. Important ingredients include parallel label propagation for both coarsening and refinement, parallel initial partitioning, a simple yet effective approach to parallel localized local search, and fast locality preserving hash tables.

Slota et al. \cite{slota2014pulp} propose PuLP (Partitioning using Label Propagation) which optimizes for multiple objective metrics simultaneously, while satisfying multiple partitioning constraints

Meyerhenke et al. \cite{meyerhenke2017parallel} propose an LPA based parallel graph partitioner. By introducing size constraints, label propagation becomes applicable for both the coarsening and the refinement phase of multilevel graph partitioning.

Bae et al. \cite{bae2020label} present a graph-partitioning algorithm based on the label propagation algorithm to improve the quality of edge cuts and achieve fast convergence. In their approach, the necessity of applying the label propagation process for all vertices is removed, and the process is applied only for candidate vertices based on a score metric. Their algorithm introduces a stabilization phase in which remote and highly connected vertices are relocated to prevent the algorithm from becoming trapped in local optima.

Das and Petrov \cite{das2011unsupervised} use label propagation for cross-lingual knowledge transfer and use the projected labels in an unsupervised part-of-speech tagger, for languages that have no labeled training data.

Aziz et al. \cite{aziz2023novel} propose a Power system sectionalizing strategy based on modified LPA.

Kozawa et al. \cite{kozawa2017gpu} propose GPU-accelerated graph clustering via parallel label propagation. They also develop algorithms to deal with large-scale datasets that do not fit into GPU memory.

Stergiou et al. \cite{stergiou2018shortcutting} Shortcutting Label Propagation for Distributed Connected Components.
}






\ignore{
Label propagation is one of the algorithms to obtain connected components in a graph \cite{esfahani2021thrifty}.

Esfahani et al. \cite{esfahani2021thrifty}
}




%% ------------------------------------------------------------------------------


\section{Preliminaries}
\label{sec:preliminaries}
Consider an undirected graph $G(V, E, w)$, where $V$ is the set of vertices, $E$ is the set of edges, and $w_{ij} = w_{ji}$ represents the weight associated with each edge. For unweighted graphs, we assume a unit weight for each edge, i.e., $w_{ij} = 1$. The neighbors of a vertex $i$ are denoted as $J_i = {j\ |\ (i, j) \in E}$, the weighted degree of each vertex as $K_i = \sum_{j \in J_i} w_{ij}$, the total number of vertices as $N = |V|$, the total number of edges as $M = |E|$, and the sum of edge weights in the undirected graph as $m = \sum_{i, j \in V} w_{ij}/2$.




\subsection{Community detection}

Disjoint community detection involves the identification of a community membership mapping, $C: V \rightarrow \Gamma$, where each vertex $i \in V$ is assigned a community ID $c \in \Gamma$, with $\Gamma$ representing the set of community IDs. We denote the vertices belonging to a community $c \in \Gamma$ as $V_c$, and the community to which a vertex $i$ belongs as $C_i$. Additionally, we denote the neighbors of vertex $i$ within community $c$ as $J_{i \rightarrow c} = {j\ |\ j \in J_i\ and\ C_j = c}$, the sum of edge weights for those neighbors as $K_{i \rightarrow c} = \sum_{j \in J_{i \rightarrow c}} w_{ij}$, the edge weight within a community $c$ as $\sigma_c = \sum_{(i, j) \in E\ and\ C_i = C_j = c} w_{ij}$, and the total edge weight of a community $c$ as $\Sigma_c = \sum_{(i, j) \in E\ and\ C_i = c} w_{ij}$ \cite{com-leskovec21}.




\subsection{Modularity}

Modularity functions as a metric for evaluating the quality of communities identified by heuristic-based community detection algorithms. It is calculated as the difference between the fraction of edges within communities and the expected fraction under random edge distribution. It lies in a range of $[-0.5, 1]$, where higher values signify superior quality \cite{com-brandes07}.\ignore{The optimization of this metric theoretically leads to the optimal grouping \cite{com-newman04, com-traag11}.} The modularity $Q$ for the identified communities is computed using Equation \ref{eq:modularity}, where $\delta$ denotes the Kronecker delta function ($\delta (x,y)=1$ if $x=y$, $0$ otherwise).\ignore{The \textit{delta modularity} of moving a vertex $i$ from community $d$ to community $c$, denoted as $\Delta Q_{i: d \rightarrow c}$, can be computed using Equation \ref{eq:delta-modularity}.}

\begin{equation}
\label{eq:modularity}
  Q
  = \frac{1}{2m} \sum_{(i, j) \in E} \left[w_{ij} - \frac{K_i K_j}{2m}\right] \delta(C_i, C_j)
  = \sum_{c \in \Gamma} \left[\frac{\sigma_c}{2m} - \left(\frac{\Sigma_c}{2m}\right)^2\right]
\end{equation}

\ignore{\begin{equation}
\label{eq:delta-modularity}
  \Delta Q_{i: d \rightarrow c}
  = \frac{1}{m} (K_{i \rightarrow c} - K_{i \rightarrow d}) - \frac{K_i}{2m^2} (K_i + \Sigma_c - \Sigma_d)
\end{equation}}




\subsection{Label Progagation Algorithm (LPA)}
\label{sec:about-rak}

LPA \cite{com-raghavan07} is a popular diffusion-based method for identifying communities of medium quality in large networks. It is simpler, faster, and more scalable compared to the Louvain method \cite{com-blondel08}. In LPA, each vertex $i$ is starts with a unique label (community id) $C_i$. During each iteration, vertices adopt the label with the highest interconnecting weight, as shown in Equation \ref{eq:lpa}. Through this iterative process, a consensus among densely connected groups of vertices is achieved, forming communities. The algorithm converges when at least $1-\tau$ fraction of vertices (where $\tau$ is the tolerance parameter) maintain their community membership. LPA exhibits a time complexity of $O(L |E|)$ and a space complexity of $O(|V| + |E|)$, with $L$ denoting the number of iterations performed \cite{com-raghavan07}. We have experimented with COPRA \cite{com-gregory10}, SLPA \cite{com-xie11}, and LabelRank \cite{com-xie13}, but found LPA to be the most performant, while yielding communities of equivalent quality \cite{sahu2023selecting}.

\begin{equation}
\label{eq:lpa}
  C_i =\ \underset{c\ \in \ \Gamma}{\arg\max} { \sum_{j \in J_i\ |\ C_j = c} w_{ij} }
\end{equation}


\section{Approach}
\label{sec:approach}
\subsection{Optimizations for LPA}
\label{sec:lpa}

We use a parallel implementation of LPA and experiment with different optimizations and parameter settings. We use the \textit{asynchronous} version of LPA, wherein threads operate independently on distinct regions of the graph. This facilitates quicker convergence, but may introduce greater variability in the final result. Further, we observe that parallel LPA obtains communities of higher quality than its sequential implementation, possibly due to randomization. We allocate a separate hashtable per thread to keep track of the total weight of each unique label linked to a vertex.

Our optimizations include using OpenMP's \verb|dynamic| loop schedule, setting an initial tolerance of $0.05$, enabling vertex pruning, employing the strict version of LPA, and using fast collision-free per-thread hashtables which are well separated in their memory addresses (\textit{Far-KV}). See below for the details on each optimization.

We evaluate multiple alternatives for each optimization, and show the relative time and the relative modularity of communities obtained by each alternative in Figure \ref{fig:rak-opt}. We perform these tests on every graph in the dataset (refer to Table \ref{tab:dataset}), conducting them five times on each graph to minimize the influence of noise. We then calculate the geometric mean for the runtime and arithmetic mean for the modularity, and represent them as ratios within each optimization category.


\subsubsection{Adjusting OpenMP loop schedule}

We attempt \textit{static}, \textit{dynamic}, \textit{guided}, and \textit{auto} loop scheduling approaches of OpenMP (each with a chunk size of $2048$) to parallelize LPA. We consider OpenMP's \verb|dynamic| loop schedule to be the best choice, due to its ability of work balancing among threads, and because it yields a runtime reduction of $27\%$ when compared to OpenMP's \textit{auto} loop schedule, while incurring only a $0.7\%$ reduction in the modularity of obtained communities (likely to be just noise).


\subsubsection{Limiting the number of iterations}

Restricting the number of iterations of LPA can ensure its termination within a reasonable number of iterations, but choosing a small limit may worsen the quality of communities obtained. Our results suggest that limiting the maximum number of iterations to $20$ strikes a good balance between runtime and modularity.


\subsubsection{Adjusting tolerance}

Using a small tolerance allows the algorithm to explore broader possibilities for community assignments, but comes at the cost of increased runtime. We find an initial tolerance of $0.05$ to be suitable. A tolerance of $0.1$ may also be acceptable, but provides a very small gain in performance when compared to a tolerance of $0.05$.


\subsubsection{Vertex pruning}

Vertex pruning is a method utilized to minimize unnecessary computation. In this approach, when a vertex alters its community, it assigns its neighbors for processing. Once a vertex has been processed, it is labeled as ineligible for further processing. However, this procedure incurs an additional overhead due to the marking and unmarking of vertices. Based on our findings, the employment of vertex pruning justifies this overhead and results in a performance enhancement of $17\%$. An illustration of vertex pruning optimization is shown in Figure \ref{fig:rak-pruning}.

\begin{figure}[hbtp]
  \centering
  \subfigure{
    \label{fig:rak-pruning--all}
    \includegraphics[width=0.78\linewidth]{out/rak-pruning.pdf}
  } \\[-2ex]
  \caption{Illustration of vertex pruning optimization: Once vertex $1$ is processed, it is unmarked. If vertex $1$ changes its community, its neighbors are marked for processing. Community membership of each vertex is indicated by border color, and marked vertices are highlighted in yellow \cite{sahu2023gvelouvain}.}
  \label{fig:rak-pruning}
\end{figure}



\subsubsection{Picking the best label}

When there exist multiple labels connected to a vertex with maximum weight, we may randomly pick one of them (non-strict LPA), or pick only the first of them (strict LPA). We implement non-strict LPA using a simple modulo operator on the label id, as we observe that using \textit{xorshift} based random number generator does not provide any advantage. Results indicate that the strict version of LPA is $1.5\times$ faster than the non-strict approach, while also offering a gain in modularity of $2.1\%$.


\subsubsection{Hashtable design}

One can utilize C++'s inbuilt map as per-thread (independent) hashtables for the LPA algorithm. However, this exhibits poor performance. Therefore, we employ a key-list and a collision-free full-size values array to dramatically improve performance. However, if the memory addresses of the hashtables are nearby (\textit{Close-KV}), even if each thread uses its own hashtable exclusively, the performance is not as high. This is possibly due to false cache-sharing. Alternatively, if we ensure that the memory address of each hashtable is farther away (\textit{Far-KV}), the performance improves. Our results indicate that \textit{Far-KV} has the best performance and is $15.8\times$ times faster than \textit{Map}, and $2.6\times$ times faster than \textit{Close-KV} with LPA. An illustration of \textit{Far-KV} hashtable is in Figure \ref{fig:rak-hashtable}.

\begin{figure}[hbtp]
  \centering
  \subfigure{
    \label{fig:rak-hashtable--all}
    \includegraphics[width=0.88\linewidth]{out/rak-hashtable.pdf}
  } \\[-2ex]
  \caption{Illustration of collision-free per-thread hashtables that are well separated in their memory addresses (Far-KV), for two threads. Each hashtable consists of a keys vector, values vector (of size $|V|$), and a key count ($N_0$/$N_1$). The value associated with each key is stored/accumulated in the index pointed by the key. As the key count of each hashtable is updated independently, we allocate it separately on the heap to avoid false cache sharing. These are used in the local-moving and aggregation phases of our rak implementation.}
  \label{fig:rak-hashtable}
\end{figure}

\begin{figure*}[hbtp]
  \centering
  \subfigure{
    \label{fig:rak-opt--all}
    \includegraphics[width=0.98\linewidth]{out/rak-opt.pdf}
  } \\[-2ex]
  \caption{Impact of\ignore{various} parameter controls and optimizations on the runtime and result quality (modularity) of LPA. We show the impact of each optimization upon the relative runtime on the left Y-axis, and upon the relative modularity on the right Y-axis.}
  \label{fig:rak-opt}
\end{figure*}





\subsection{Our optimized LPA implementation}

We now explain the implementation of GVE-LPA in Algorithm \ref{alg:rak}. Its main step is the \texttt{lpa()} function (lines \ref{alg:rak--main-begin}-\ref{alg:rak--main-end}), which takes the input graph $G$ and assigns community memberships (or labels) $C$ to each vertex. In lines \ref{alg:rak--init-begin}-\ref{alg:rak--init-end}, we first initialize the labels $C$ of each vertex in $G$, and mark all vertices as unprocessed. We then perform iterations to propagate labels based on the weighted influence of neighboring vertices, limited to $MAX\_ITERATIONS$ (lines \ref{alg:rak--iters-begin}-\ref{alg:rak--iters-end}). In each iteration, we invoke the \texttt{lpaMove()} function to perform label propagation, and count the number of nodes with updated labels $\Delta N$ (line \ref{alg:rak--propagate}). If the ratio of $\Delta N$ to the total number of nodes $N$ is within a specified tolerance $\tau$, convergence has been achieved, and we terminate the loop (line \ref{alg:rak--converged}). Upon completing all iterations, we return the final labels $C$ (line \ref{alg:rak--main-return}).

\begin{algorithm}[hbtp]
\caption{GVE-LPA: Our parallel Label Propagation Algorithm (LPA).}
\label{alg:rak}
\begin{algorithmic}[1]
\Require{$G'$: Input graph}
\Require{$C$: Community membership of each vertex}
\Require{$G'$: Input/super-vertex graph}
\Require{$C'$: Community membership of each vertex in $G'$}
\Require{$K'$: Total edge weight of each vertex}
\Require{$\Sigma'$: Total edge weight of each community}
\Ensure{$G'_{C'}$: Community vertices (CSR)}
\Ensure{$H_t$: Collision-free per-thread hashtable}
\Ensure{$l_i$: Number of iterations performed (per pass)}
\Ensure{$l_p$: Number of passes performed}
\Ensure{$\tau$: Per iteration tolerance}
\Ensure{$\tau_{agg}$: Aggregation tolerance}

\Statex

\Function{louvain}{$G$} \label{alg:louvain--begin}
  \State Vertex membership: $C \gets [0 .. |V|)$ \textbf{;} $G' \gets G$ \label{alg:louvain--initialization}
  \ForAll{$l_p \in [0 .. \text{\small{MAX\_PASSES}})$} \label{alg:louvain--passes-begin}
    \State $\Sigma' \gets K' \gets vertexWeights(G')$ \textbf{;} $C' \gets [0 .. |V'|)$ \label{alg:louvain--reset-weights}
    \State Mark all vertices in $G'$ as unprocessed \label{alg:louvain--reset-affected}
    \State $l_i \gets louvainMove(G', C', K', \Sigma')$ \label{alg:louvain--local-move}
    \If{$l_i \le 1$} \textbf{break} \Comment{Globally converged?} \label{alg:louvain--globally-converged}
    \EndIf
    \State $|\Gamma|, |\Gamma_{old}| \gets$ Number of communities in $C$, $C'$
    \If{$|\Gamma|/|\Gamma_{old}| > \tau_{agg}$} \textbf{break} \Comment{Low shrink?} \label{alg:louvain--aggregation-tolerance}
    \EndIf
    \State $C' \gets$ Renumber communities in $C'$ \label{alg:louvain--renumber}
    \State $C \gets$ Lookup dendrogram using $C$ to $C'$ \label{alg:louvain--lookup}
    \State $G' \gets louvainAggregate(G', C')$ \label{alg:louvain--aggregate}
    \State $\tau \gets \tau / \text{\small{TOLERANCE\_DROP}}$ \Comment{Threshold scaling} \label{alg:louvain--threshold-scaling}
  \EndFor \label{alg:louvain--passes-end}
  \State $C \gets$ Lookup dendrogram using $C$ to $C'$ \label{alg:louvain--lookup-last}
  \Return{$C$} \label{alg:louvain--return}
\EndFunction \label{alg:louvain--end}
\end{algorithmic}
\end{algorithm}


The \texttt{lpaMove()} function (lines \ref{alg:rak--move-begin}-\ref{alg:rak--move-end}) iterates over unprocessed vertices in parallel. For each unprocessed vertex $i$ in the graph $G$, we mark $i$ as processed - vertex pruning (line \ref{alg:rak--prune}), obtain the total edge weight of connected labels in per-thread hashtable $H_t$ with the \texttt{scanCommunities()} function \ref{alg:rak--scan}, and select the most weighted label $c^*$ (line \ref{alg:rak--best-community}). If $c^*$ is not the same as the current label of $i$, we update the label of $i$, increment the count of changed vertices $\Delta N$, and mark the neighbors of $i$ as unprocessed for the next iteration (lines \ref{alg:rak--perform-move}-\ref{alg:rak--remark}). After having processed all vertices, we return the total number of vertices with updated labels $\Delta N$ (line \ref{alg:rak--move-return}). The \texttt{scanCommunities()} (lines \ref{alg:rak--scan-begin}-\ref{alg:rak--scan-end}) iterates over the neighbors of the current vertex $i$, excluding itself, and calculates the total weight of each label in the hashtable $H_t$.


\section{Evaluation}
\label{sec:evaluation}
\subsection{Experimental Setup}
\label{sec:setup}

We use a server that has two $16$-core x86-based Intel Xeon Gold 6226R processors running at $2.90$ GHz. Each core has an L1 cache of $1$ MB, an L2 cache of $16$ MB, and a shared L3 cache of $22$ MB. The machine has $93.4$ GB of system memory and runs on CentOS Stream 8. We use GCC 8.5 and OpenMP 4.5. Table \ref{tab:dataset} shows the graphs we use in our experiments. All of them are obtained from the SuiteSparse Matrix Collection \cite{suite19}.

\begin{table}[hbtp]
  \centering
  \caption{List of $13$ graphs obtained from the SuiteSparse Matrix Collection \cite{suite19} (directed graphs are marked with $*$). Here, $|V|$ is the number of vertices, $|E|$ is the number of edges (after adding reverse edges), $D_{avg}$ is the average degree, and $|\Gamma|$ is the number of communities identified with GVE-LPA.\ignore{In the table, B refers to a billion, M refers to a million and K refers a thousand.}}
  \label{tab:dataset}
  \begin{tabular}{|c||c|c|c|c|}
    \toprule
    \textbf{Graph} &
    \textbf{\textbf{$|V|$}} &
    \textbf{\textbf{$|E|$}} &
    \textbf{\textbf{$D_{avg}$}} &
    \textbf{\textbf{$|\Gamma|$}} \\
    % \textbf{$1 - \Gamma_G$} \\
    \midrule
    \multicolumn{5}{|c|}{\textbf{Web Graphs (LAW)}} \\ \hline
    indochina-2004$^*$ & 7.41M & 341M & 41.0 & 147K \\ \hline  % & \num{4.7e-4} & 2.9 GB
    uk-2002$^*$ & 18.5M & 567M & 16.1 & 383K \\ \hline  % & \num{9.6e-5} & 16 GB
    arabic-2005$^*$ & 22.7M & 1.21B & 28.2 & 213K \\ \hline  % & \num{5.5e-4} & 11 GB
    uk-2005$^*$ & 39.5M & 1.73B & 23.7 & 677K \\ \hline  % & \num{9.6e-5} & 16 GB
    webbase-2001$^*$ & 118M & 1.89B & 8.6 & 6.48M \\ \hline  % & \num{7.3e-7} & 18 GB
    it-2004$^*$ & 41.3M & 2.19B & 27.9 & 611K \\ \hline  % & \num{3.8e-4} & 19 GB
    sk-2005$^*$ & 50.6M & 3.80B & 38.5 & 284K \\ \hline  % & \num{5.8e-4} & 33 GB
    \multicolumn{5}{|c|}{\textbf{Social Networks (SNAP)}} \\ \hline
    com-LiveJournal & 4.00M & 69.4M & 17.4 & 2.19K \\ \hline  % & \num{7.9e-4} & 480 MB
    com-Orkut & 3.07M & 234M & 76.2 & 49 \\ \hline  % & \num{6.7e-2} & 1.7 GB
    \multicolumn{5}{|c|}{\textbf{Road Networks (DIMACS10)}} \\ \hline
    asia\_osm & 12.0M & 25.4M & 2.1 & 278K \\ \hline  % & \num{8.4e-4} & 200 MB
    europe\_osm & 50.9M & 108M & 2.1 & 1.52M \\ \hline  % & \num{6.6e-4} & 910 MB
    \multicolumn{5}{|c|}{\textbf{Protein k-mer Graphs (GenBank)}} \\ \hline
    kmer\_A2a & 171M & 361M & 2.1 & 40.1M \\ \hline  % & \num{9.4e-5} & 3.2 GB
    kmer\_V1r & 214M & 465M & 2.2 & 48.7M \\ \hline  % & \num{3.2e-4} & 4.2 GB
  \bottomrule
  \end{tabular}
\end{table}
% We convert directed graphs (marked with $*$) to undirected by duplicating edges in the reverse direction, and set the weight of each edge to $1$. and $F_{size}$ is size of the \textit{MatrixMarket} file

\begin{figure*}[hbtp]
  \centering
  \subfigure[Runtime in seconds (logarithmic scale) with \textit{igraph}, \textit{FLPA}, \textit{NetworKit}, and \textit{GVE-LPA}]{
    \label{fig:rak-compare--runtime}
    \includegraphics[width=0.98\linewidth]{out/rak-runtime.pdf}
  } \\[-0ex]
  \subfigure[Speedup of \textit{GVE-LPA} with respect to \textit{igraph}, \textit{FLPA}, \textit{NetworKit}.]{
    \label{fig:rak-compare--speedup}
    \includegraphics[width=0.98\linewidth]{out/rak-speedup.pdf}
  } \\[-0ex]
  \subfigure[Modularity of communities obtained with \textit{igraph}, \textit{FLPA}, \textit{NetworKit}, and \textit{GVE-LPA}.]{
    \label{fig:rak-compare--modularity}
    \includegraphics[width=0.98\linewidth]{out/rak-modularity.pdf}
  } \\[-2ex]
  \caption{Runtime in seconds (logarithmic scale), speedup, and modularity of communities obtained with \textit{igraph}, \textit{FLPA}, \textit{NetworKit}, and \textit{GVE-LPA} for each graph in the dataset.}
  \label{fig:rak-compare}
\end{figure*}





\subsection{Comparing Performance of GVE-LPA}

We now compare the performance of GVE-LPA with igraph LPA, FLPA, and NetworKit LPA. For Vite, we convert the graph datasets to Vite's binary graph format, run it on a single node\ignore{(Vite supports distributed community detection)} with threshold cycling/scaling optimization, and measure the reported average total time. For Grappolo, we measure the run it on the same system, and measure the reported total time. For NetworKit, we use a Python script to invoke \texttt{PLM} (Parallel Louvain Method), and measure the total time reported with \texttt{getTiming()}. For each graph, we measure the runtime of each implementation five times, for averaging. We also record the modularity of communities obtained, as reported by each implementation.

Figure \ref{fig:rak-compare--runtime} shows the runtimes of Vite (Louvain), Grappolo (Louvain), NetworKit Louvain, and GVE-Louvain on each graph in the dataset. Figure \ref{fig:rak-compare--speedup} shows the speedup of GVE-Louvain with respect to each implementation mentioned above. GVE-Louvain is on average $50\times$, $22\times$, and $20\times$ faster than Vite, Grappolo, and NetworKit respectively. On the \textit{sk-2005} graph, GVE-Louvain finds communities in $6.8$ seconds, and thus achieve a processing rate of $560$ million edges/s. Figure \ref{fig:rak-compare--modularity} shows the modularity of communities obtained with each implementation. GVE-Louvain on average obtains $3.1\%$ higher modularity than Vite (especially on web graphs), and $0.6\%$ lower modularity than Grappolo and NetworKit (especially on social networks with poor clustering).

\begin{figure}[hbtp]
  \centering
  \subfigure{
    \label{fig:rak-hardness--all}
    \includegraphics[width=0.98\linewidth]{out/rak-hardness.pdf}
  } \\[-2ex]
  \caption{Runtime $/ |E|$ factor with \textit{GVE-LPA} for each graph in the dataset.}
  \label{fig:rak-hardness}
\end{figure}

\begin{figure}[hbtp]
  \centering
  \includegraphics[width=0.98\linewidth]{out/rak-ss.pdf} \\[0ex]
  \caption{Average speedup of \textit{GVE-Louvain} and \textit{GVE-LPA} with increasing number of threads (in multiples of 2).}
  \label{fig:rak-ss}
\end{figure}





\subsection{Analyzing Performance of GVE-LPA}

The phase-wise and pass-wise split of GVE-Louvain is shown in Figures \ref{fig:louvain-splits--phase} and \ref{fig:louvain-splits--pass} respectively. Figure \ref{fig:louvain-splits--phase} indicates that GVE-Louvain spends most of the runtime in the local-moving phase on \textit{web graphs}, \textit{road networks}, and \textit{protein k-mer graphs}, while it devotes majority of the runtime in the aggregation phase on \textit{social networks}. The pass-wise split (Figure \ref{fig:louvain-splits--pass}) indicates that the first pass dominates runtime on high-degree graphs (\textit{web graphs} and \textit{social networks}), while subsequent passes prevail in execution time on low-degree graphs (\textit{road networks} and \textit{protein k-mer graphs}).

On average, $49\%$ of GVE-Louvain's runtime is spent in the local-moving phase, $35\%$ is spent in the aggregation phase, and $16\%$ is spent in other steps (initialization, renumbering communities, looking up dendrogram, and resetting communities) of the algorithm. Further, $67\%$ of the runtime is spent in the first pass of the algorithm, which is the most expensive pass due to the size of the original graph (later passes work on super-vertex graphs) \cite{com-wickramaarachchi14}.

We also observe that graphs with lower average degree (\textit{road networks} and \textit{protein k-mer graphs}) and graphs with poor community structure (such as \verb|com-LiveJournal| and \verb|com-Orkut|) have a larger $\text{runtime}/|E|$ factor, as shown in Figure \ref{fig:rak-hardness}.





\subsection{Strong Scaling of GVE-LPA}

Finally, we measure the strong scaling performance of GVE-Louvain. To this end, we adjust the number of threads from $1$ to $64$ in multiples of $2$ for each input graph, and measure the time taken for finding communities with GVE-Louvain (five times for averaging). The results are shown in Figure \ref{fig:rak-ss}. With 32 threads, GVE-Louvain obtains a $10.4\times$ speedup compared to running with a single thread, i.e., its performance increases by $1.6\times$ for every doubling of threads. Scaling is limited due to the various sequential steps/phases in the algorithm. At 64 threads, GVE-Louvain is impacted by NUMA effects, and offer speedups of only $11.4\times$.


\section{Conclusion}
\label{sec:conclusion}
In summary, this study focused on optimizing the Label Propagation Algorithm (LPA), a high-speed community detection algorithm, in the shared memory setting.\ignore{We considered 6 different optimizations, which significantly improve the performance of the algorithm.} Comparative assessments against competitive open-source implementations (FLPA\ignore{\cite{traag2023large}}) and packages (igraph\ignore{\cite{csardi2006igraph}} and NetworKit\ignore{\cite{staudt2016networkit}}) indicate that GVE-LPA outperforms FLPA, igraph LPA, and NetworKit LPA by $139\times$, $97,000\times$, and $40\times$ respectively, while identifying communities of $6.6\%$ / $0.2\%$ higher quality\ignore{(modularity)} than FLPA and igraph LPA respectively, and of $4.1\%$ lower quality than NetworKit LPA. On a web graph with $3.8$ billion edges, GVE-LPA identifies communities in $2.7$ seconds, and thus achieves a processing rate of $1.4$ billion edges/s. GVE-LPA is ideal for applications requiring a high-speed clustering algorithm while accepting a compromise in clustering quality, as it is on average $5.4\times$ faster than GVE-Louvain\ignore{\cite{sahu2023gvelouvain}}, has a strong scaling factor of $1.7\times$ for every doubling of threads, but identifies communities with on average $10.9\%$ lower modularity than GVE-Louvain.\ignore{Future research could explore dynamic algorithms for LPA to accommodate evolving graphs in real-world scenarios. This would allow interactive updation of community memberships of vertices on large graphs.}


%% The acknowledgments section.
\begin{acks}
I would like to thank Prof. Kishore Kothapalli, Prof. Dip Sankar Banerjee, Balavarun Pedapudi, Souvik Karfa, and Vincent Traag for their support.
\end{acks}

%% Bibliography style to be used, and the bibliography file.
\bibliographystyle{ACM-Reference-Format}
\bibliography{main}

\end{document}
\endinput
%% End of file.




%% NOTES:
%% - Parallelization seems to be not efficient for small batch updates.
%% - Discuss about conflicting updates
%% - 


%% TODO:
%% - Scale up the size of the graphs
%% - Move experiments to a better server
%% - Include a weak- and strong- scalabiilty plot: run the expt from 2 to 128 threads
%% - overall space planning
%% - add a few lines on novelty of the paper.
%% - table comparison of related work
%% - Include a section on preliminaries that talks about the various algorithmic ideas (Louvain, Label Propagation)

%% Workplan:
%% - KK -- Read Introduction, Related Work,
%% - Dip Sankar -- Approach -- summarize the main algorithmic ideas,
%% - Subhajit -- Results -- Plots, scalability, Dataset, experiments, implementation details,
