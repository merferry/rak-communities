In summary, this study focuses on optimizing the Label Propagation Algorithm (LPA), a high-speed community detection algorithm, in the shared memory setting. We consider 6 different optimizations, which significantly improve the performance of the algorithm. Comparative assessments against competitive open-source implementations (FLPA\ignore{\cite{traag2023large}}) and packages (igraph\ignore{\cite{csardi2006igraph}} and NetworKit\ignore{\cite{staudt2016networkit}}) indicate that GVE-LPA outperforms FLPA, igraph LPA, and NetworKit LPA by $139\times$, $97000\times$, and $40\times$ respectively, while identifying communities of $6.6\%$ / $0.2\%$ higher quality\ignore{(modularity)} than FLPA and igraph LPA respectively, and of $4.1\%$ lower quality than NetworKit LPA.

On a web graph with $3.8$ billion edges, GVE-LPA identifies communities in $2.7$ seconds, and thus achieves a processing rate of $1.4$ billion edges/s. GVE-LPA is thus ideal for applications requiring a high-speed clustering algorithm while accepting a compromise in clustering quality --- it is on average $5.4\times$ faster than GVE-Louvain\ignore{\cite{sahu2023gvelouvain}}, and has a strong scaling factor of $1.7\times$ for every doubling of threads, but identifies communities with on average $10.9\%$ lower modularity than GVE-Louvain.\ignore{Future research could explore dynamic algorithms for LPA to accommodate evolving graphs in real-world scenarios. This would allow interactive updation of community memberships of vertices on large graphs.}
