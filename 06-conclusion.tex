% In summary, this study concentrates on optimizing the Label Propagation Algorithm (LPA), a high-speed community detection algorithm, in the shared memory setting. The implementation, named GVE-LPA, undergoes six distinct optimizations, significantly enhancing its performance. Comparative assessments against competitive open-source implementations (FLPA) and packages (igraph and NetworKit) reveal that GVE-LPA, running on a server with dual 16-core Intel Xeon Gold 6226R processors, outperforms FLPA, igraph LPA, and NetworKit LPA by average speedups of $97,000\times$, $118,000\times$, and $40\times$, respectively. GVE-LPA also achieves higher modularity, averaging $0.6%$ and $0.2%$ more than FLPA and igraph LPA, respectively, but with a $4.1%$ lower modularity than NetworKit LPA. On a web graph with $3.8$ billion edges, GVE-LPA identifies communities in $2.7$ seconds, achieving a processing rate of $1.4$ billion edges/s. While GVE-LPA is ideal for applications requiring a high-speed clustering algorithm, accepting a compromise in clustering quality, it is, on average, $5.4\times$ faster than GVE-Louvain \cite{sahu2023gvelouvain}, with an average $10.9%$ lower modularity. Additionally, GVE-LPA exhibits a strong scaling factor of $1.7\times$ for every doubling of threads. Future research could explore dynamic algorithms for LPA to accommodate evolving graphs in real-world scenarios, facilitating interactive updates of community memberships for vertices over time.

In summary, this study concentrates on optimizing the Label Propagation Algorithm (LPA), a high-speed community detection algorithm, in the shared memory setting. We consider 6 different optimizations, which significantly improve the performance of the algorithm. Comparative assessments against competitive open-source implementations (FLPA\ignore{\cite{traag2023large}}) and packages (igraph\ignore{\cite{csardi2006igraph}} and NetworKit\ignore{\cite{staudt2016networkit}}) indicate that GVE-LPA outperforms FLPA, igraph LPA, and NetworKit LPA by $118,000\times$, $97,000\times$, and $40\times$, respectively, while identifying communities of roughly the same quality as FLPA and igraph LPA, and of $4.1\%$ lower quality (modularity) than NetworKit LPA.

On a web graph with $3.8$ billion edges, GVE-LPA identifies communities in $2.7$ seconds, and thus achieves a processing rate of $1.4$ billion edges/s. GVE-LPA is ideal for applications requiring a high-speed clustering algorithm while accepting a compromise in clustering quality, as it is on average $5.4\times$ faster than GVE-Louvain\ignore{\cite{sahu2023gvelouvain}}, has a strong scaling factor of $1.7\times$ for every doubling of threads, but identifies communities with on average $10.9\%$ lower modularity than GVE-Louvain. Future research could explore dynamic algorithms for LPA to accommodate evolving graphs in real-world scenarios. This would allow interactive updation of community memberships of vertices on large graphs.
