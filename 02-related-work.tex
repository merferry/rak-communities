The \textit{Label Propagation Algorithm (LPA)} is a diffusion-based technique for identifying communities. It is faster and more scalable than Louvain, as it does not require repeated optimization steps and is easier to parallelize \cite{com-newman04, com-raghavan07}. For parallelization of LPA, vertex assignment has been achieved with guided scheduling \cite{staudt2015engineering}, parallel bitonic sort \cite{soman2011fast}, and pre-partitioning of the graph \cite{kuzmin2015parallelizing}.

\ignore{LPA considers only adjacent nodes for label selection. Alternative neighborhood definitions for LPA consider nodes at a distance less than or equal to 2 \cite{lou2013detecting, chen2017detecting} or use a tunable parameter for distance \cite{sun2014label}.}In terms of quality, LPA may generate communities with low modularity, leading to the emergence of a dominant community that engulfs the majority of nodes, obstructing finer community structures \cite{com-gregory10}. To address instability of result, community structure quality, and performance due to vertex processing order, specific ordering strategies have been attempted. These include updating only active nodes \cite{xie2011community} or prioritizing certain subsets based on properties like being core or boundary nodes \cite{gui2018community}. In some cases, oscillations in label assignments may occur instead of convergence. To address this, asynchronous mode \cite{leung2009towards}, alternate updates of independent node subsets \cite{cordasco2012label}, and parallel graph coloring techniques \cite{cordasco2012label} have been attempted. Raghavan et al. \cite{com-raghavan07} report that after only five iterations, labels of $95\%$ of nodes converge to their final values.

Additional improvements upon the LPA include using a stable (non-random) mechanism of label choosing in the case of multiple best labels \cite{com-xing14}, addressing the issue of monster communities \cite{com-berahmand18, com-sattari18}, identifying central nodes and combining communities for improved modularity \cite{com-you20}, and using frontiers with alternating push-pull to reduce the number of edges visited and improve solution quality \cite{com-liu20}. A number of variants of LPA have been proposed, but the original formulation is still the simplest and most efficient \cite{garza2019community}.

A few open source implementations and software packages have been developed for community detection using LPA. Fast Label Propagation Algorithm (FLPA) \cite{traag2023large} is a fast variant of the LPA, which utilizes a queue-based approach to process only vertices with recently updated neighborhoods. NetworKit \cite{staudt2016networkit} is a software package designed for analyzing the structural aspects of graph data sets with billions of connections. It is implemented as a hybrid with C++ kernels and a Python frontend, and includes parallel implementation of LPA. igraph \cite{csardi2006igraph} is a similar package, written in C, with Python, R, and Mathematica frontends. It is widely used in academic research, and includes implementation of LPA.






% https://www.sciencedirect.com/science/article/abs/pii/S0378437119312026
% http://proceedings.mlr.press/v32/fujiwara14.pdf
