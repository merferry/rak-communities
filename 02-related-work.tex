The \textit{Label Propagation Algorithm (LPA)} is a diffusion-based technique for identifying communities. It is faster and more scalable than Louvain, as it does not require repeated optimization steps and is easier to parallelize \cite{com-newman04, com-raghavan07}. Improvements upon the LPA include using a stable (non-random) mechanism of label choosing in the case of multiple best labels \cite{com-xing14}, addressing the issue of monster communities \cite{com-berahmand18, com-sattari18}, identifying central nodes and combining communities for improved modularity \cite{com-you20}, and using frontiers with alternating push-pull to reduce the number of edges visited and improve solution quality \cite{com-liu20}.

A few open source implementations and software packages have been developed for community detection using LPA. Fast Label Propagation Algorithm (FLPA) \cite{traag2023large} is a fast variant of the LPA, which utilizes a queue-based approach to process only vertices with recently updated neighborhoods. NetworKit \cite{staudt2016networkit} is a software package designed for analyzing the structural aspects of graph data sets with billions of connections. It is implemented as a hybrid with C++ kernels and a Python frontend, and includes parallel implementation of LPA. igraph \cite{csardi2006igraph} is a similar package, written in C, with Python, R, and Mathematica frontends. It is widely used in academic research, and includes implementation LPA.

The \textit{Label Propagation Algorithm (LPA)} is a diffusion-based technique for identifying communities. It is faster and more scalable than Louvain, as it does not require repeated optimization steps and is easier to parallelize \cite{com-newman04, com-raghavan07}. Improvements upon the LPA include using a stable (non-random) mechanism of label choosing in the case of multiple best labels \cite{com-xing14}, addressing the issue of monster communities \cite{com-berahmand18, com-sattari18}, identifying central nodes and combining communities for improved modularity \cite{com-you20}, and using frontiers with alternating push-pull to reduce the number of edges visited and improve solution quality \cite{com-liu20}.

The \textit{Label Propagation Algorithm (LPA)} is a diffusion-based technique for identifying communities. It is faster and more scalable than Louvain, as it does not require repeated optimization steps and is easier to parallelize \cite{com-newman04, com-raghavan07}. Improvements upon the LPA.


% https://www.sciencedirect.com/science/article/abs/pii/S0378437119312026
% http://proceedings.mlr.press/v32/fujiwara14.pdf
